\section{Math}

\subsection{Euler Function}
\lstinputlisting{"./template/math/euler.c"}

\subsection{Möbius Function}
\lstinputlisting{"./template/math/mobius.cc"}

\subsection{Number Theory Inverse}
\lstinputlisting{"./template/math/inv.cc"}

\subsection{Chinese Remainder Theorem}
\[
	x \equiv a_i \quad (\mathrm{mod}\;m_i)
\]
\lstinputlisting{"./template/math/crt.c"}

\subsection{Linear congruences}
\lstinputlisting{"./library/math/mod_equ.cc"}
Usage:
For
\[
\left\{
    \begin{array}{l}
    x \equiv a_1 \mod m_1 \\
    x \equiv a_2 \mod m_2 \\
    \vdots \\
    x \equiv a_k \mod m_k \\
    \end{array}
\right.
\]
run
\begin{lstlisting}
mod_equ_resolver solver;
for (int i = 1; i <= k; i++)
    solver.onemore(a[i], m[i]);
\end{lstlisting}
then the solution is
\[
    x \equiv solver.a \mod solver.m
\]

\subsection{FFT}
\lstinputlisting{"./template/math/fft.cc"}

\subsection{NTT}
\lstinputlisting{"./template/math/ntt.cc"}

\subsection{Fast Walsh–Hadamard transform}
\begin{itemize}
\item 异或
$$
\mathcal{F}\{A\} =
\left[\mathcal{F}\{A_0\}+\mathcal{F}\{A_1\},
\mathcal{F}\{A_0\}-\mathcal{F}\{A_1\}\right]
$$
$$
\mathcal{F}^{-1}\{A\} =
\left[
\mathcal{F}^{-1}\{\frac{A_0+A_1}{2}\},
\mathcal{F}^{-1}\{\frac{A_0-A_1}{2}\}
\right]
$$
\item 按位与
$$
\mathcal{F}\{A\} =
\left[\mathcal{F}\{A_0\}+\mathcal{F}\{A_1\},
\mathcal{F}\{A_1\}\right]
$$
$$
\mathcal{F}^{-1}\{A\} =
\left[
\mathcal{F}^{-1}\{A_0\}-\mathcal{F}^{-1}\{A_1\},
\mathcal{F}^{-1}\{A_1\}
\right]
$$
\item 按位或
$$
\mathcal{F}\{A\} =
\left[\mathcal{F}\{A_0\},
\mathcal{F}\{A_1\}+\mathcal{F}\{A_0\}\right]
$$
$$
\mathcal{F}^{-1}\{A\} =
\left[
\mathcal{F}^{-1}\{A_0\},
\mathcal{F}^{-1}\{A_1\}-\mathcal{F}^{-1}\{A_0\}
\right]
$$
\end{itemize}
\lstinputlisting{"./template/math/fwt.cc"}

\subsection{Lucas}
\lstinputlisting{"./library/math/lucas.cc"}

\subsection{Linear Programming}
\lstinputlisting{"./template/math/lp.cc"}

\subsection{Big Prime Test}
\lstinputlisting{"./template/math/prime-test.cc"}
\subsubsection{Miller Rabin}
\lstinputlisting{"./library/math/miller-rabin.cc"}
\subsubsection{Pollard's rho}
\lstinputlisting{"./library/math/pollards-rho.cc"}

\subsection{Montgomery modular multiplication}
\lstinputlisting{"./library/math/montgomery.cc"}

\subsection{Berlekamp Massey}
\lstinputlisting{"./library/math/berlekamp-massey.cc"}

\subsection{Inclusion–exclusion principle}
\subsubsection{General form}
若 $A_1, A_2, \dots, A_n$ 为有限集,则
$$
\left|\bigcup_{i=1}^n{A_i}\right|
=
\sum_{\phi \ne J \subseteq \{1,2,\dots,n\}}{(-1)^{ |J| - 1 }\left|\bigcap_{j \in J}{A_j}\right| }
$$
\subsubsection{A generalization}

若
$$ g(A) = \sum_{S \subseteq A}{f(S)} $$
则
$$ f(A) = \sum_{S \subseteq A}{(-1)^{ |A| - |S| }g(S)} $$

更一般的,如果 $S$ 是多重集合(multiset),那么
$$ f(A) = \sum_{S \subseteq A}{\mu(A - S)g(S)}$$
其中
\begin{itemize}
\item 当 $S$ 是含有偶数个元素的集合(没有重复元素)时,$\mu(S) = 1$
\item 当 $S$ 是含有奇数个元素的集合(没有重复元素)时,$\mu(S) = -1$
\item 当 $S$ 含有重复元素时,$\mu(S) = 0$.
\end{itemize}
\subsubsection{Applications}
\begin{enumerate}
\item 乱序排列

    如果集合 $A$ 含有 $n$ 个元素,
    则乱序排列的数目为 $[n! / e]$, $[x]$ 表示最接近 $x$ 的整数.
\end{enumerate}

\subsection{Lindström–Gessel–Viennot lemma}
对于一张无边权的 DAG 图,
给定 $n$ 个起点和对应的 $n$ 个终点,
这 $n$ 条不相交路径的方案数为
\[
\left|
\begin{array}{cccc}
e(a_1, b_1) & e(a_1, b_2) & \cdots & e(a_1, b_n) \\
e(a_2, b_1) & e(a_2, b_2) & \cdots & e(a_2, b_n) \\
\vdots & \vdots & \ddots & \vdots \\
e(a_n, b_1) & e(a_n, b_2) & \cdots & e(a_n, b_n)
\end{array}
\right|
\quad (\text{该矩阵的行列式})
\]
其中 $e(a,b)$ 为图上 $a$ 到 $b$ 的方案数.
