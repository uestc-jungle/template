\subsection{C++ Reference}

\subsubsection{Debug}
\begin{lstlisting}
#ifdef JDEBUGMODE
#define jdebug(format, ...) fprintf(stderr, "L%d: " format, __LINE__, __VA_ARGS__)
#define jshow(...) fprintf(stderr, "%s\n", #__VA_ARGS__)
#else
#define jdebug(...) (void (0))
#define jshow(...) (void (0))
#endif
\end{lstlisting}

\subsubsection{STL}

\textbf{bitset}
\begin{lstlisting}
template <size_t N> class bitset;
\end{lstlisting}

access
\begin{description}
    \item[count] Count bits set
    \item[test] Return bit value
    \item[any] Test if any bit is set
    \item[none] Test if no bit is set
    \item[all] Test if all bits are set
\end{description}
operations
\begin{description}
    \item[set] Set bits 1
    \item[reset] Reset bits 0
    \item[flip] Flip bits
    \item[to\_string]
    \item[to\_ullong] to unsigned long long
\end{description}

could be used for unordered\_set or unordered\_map
\begin{lstlisting}
template <class T> struct hash;              // unspecialized
template <size_t N> struct hash<bitset<N>>;  // bitset specializatio
\end{lstlisting}

\textbf{unordered\_map}
相对于 \verb|map|,插入慢,查询快.

\textbf{multiset}
对于 \verb|erase| 操作,
如果传值进去,会把所有该值的元素删去;
如果传迭代器,则只删一个元素。

\textbf{other}
\begin{lstlisting}
template <class InputIterator1, class InputIterator2>
  bool includes ( InputIterator1 first1, InputIterator1 last1,
                  InputIterator2 first2, InputIterator2 last2 );

template <class InputIterator1, class InputIterator2, class Compare>
  bool includes ( InputIterator1 first1, InputIterator1 last1,
                  InputIterator2 first2, InputIterator2 last2, Compare comp );

auto il = { 10, 20, 30 };  // the type of il is an initializer_list 

\end{lstlisting}
