\section{Graph}

\subsection{Tree}
\subsubsection{Universe}
\lstinputlisting{"./template/graph/tree.cc"}

\subsubsection{Point Divide and Conquer}

Version 1
\lstinputlisting{"./template/graph/tree/point-divide-and-conquer.cc"}

Version 2
\lstinputlisting{"./template/graph/tree/point-divide-and-conquer-2.cc"}

\subsubsection{Hevay chain decompostion}
\lstinputlisting{"./template/graph/tree/link-cut-tree.cc"}

\subsection{2-SAT}
\lstinputlisting{"./template/graph/2-sat.cc"}

\subsection{Cut Edge and Point}
\lstinputlisting{"./template/graph/cut.cc"}

\subsection{Euler Path}
\lstinputlisting{"./template/graph/euler.c"}

\subsection{Shortest Path}
\subsubsection{Dijkstra}
\lstinputlisting{"./template/graph/dijstra.cc"}
\subsubsection{Shortest Path Fast Algorithm}
\lstinputlisting{"./template/graph/spfa.cc"}
\subsubsection{$K$-th shortest path}
\lstinputlisting{"./template/graph/kth-sp.cc"}

\subsection{Maxflow}
\lstinputlisting{"./template/graph/maxflow.cc"}

\subsection{Strongly Connected Component}
\lstinputlisting{"./template/graph/scc.cc"}

\subsection{Perfect elimination ordering}
求弦图的最大团数/最小色数的时候,只要在完美消除序列上从后往前贪心染色即可。
 
而求最大独立集/最小团覆盖的时候,只要在完美消除序列上从前往后贪心取点即可。
\lstinputlisting{"./template/graph/perfect-elimination-ordering.cc"}
