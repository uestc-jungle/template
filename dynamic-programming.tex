\section{Dynamic Programming}

\subsection{LIS $O(n\log{n})$}
\lstinputlisting{"./template/dp/lis.cc"}

\subsection{LCS $O(n\log{n})$}

总的来说,就是把 LCS 转化成 LIS,然后用 LIS 的 $\mathcal{O}(N\log{N})$ 算法来求解。

实现如下:(引用)

假设有两个序列 $s_1[1\dots{6}] = abcadc$, $s_2[1\dots{7}] = cabedab$.

记录 $s_1$ 中每个元素在 $s_2$ 中出现的位置, 再将位置按降序排列, 则上面的例子可表示为:

$loc(a) = \{ 6, 2 \}$,
$loc( b ) = \{ 7, 3 \}$,
$loc( c ) = \{ 1 \}$, 
$loc( d ) = \{ 5 \}$.
(倒着扫一遍 $s_2$ 即可把位置扔进 \verb|vector|).

将 $s_1$ 中每个元素的位置按 $s_1$ 中元素的顺序排列成一个序列
$s_3 = \{ 6, 2, 7, 3, 1, 6, 2, 5, 1 \}$.

在对 $s_3$ 求LIS得到的值即为求LCS的答案。


\subsection{Improved by quadrilateral inequality}
\lstinputlisting{"./template/dp/inequ.cc"}

\subsection{Improved by Slope}
\lstinputlisting{"./template/dp/slope.cc"}

\subsection{Steiner Tree}
令 $f[i][sta]$ 表示 $i$ 号节点,
与其他节点的连通性为 $sta$ 时的最小代价,
这里 $sta$ 是一个二进制数,
在它二进制下的每一位中,$0$ 表示不连通,$1$ 表示联通

状态转移:
\begin{itemize}
    \item 由子集转移而来
    \[
        f[i][sta] = \min_{s \subseteq sta}\{f[i][s] + f[i][sta \setminus s] - val[i]\}
    \]
    \item 由不含该节点的状态转移而来
    \[
        f[i][j] = \min\{f[k][j] + val[i]\}
    \]
\end{itemize}

\lstinputlisting{"./template/dp/steiner-tree.cc"}
